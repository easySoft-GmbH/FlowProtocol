\documentclass[12pt ,a4paper]{article}
\usepackage[ansi]{umlaute}
\usepackage[german]{babel}
\usepackage{fancyheadings}
\usepackage{geometry}
\title{Softwareentwicklung mit FlowProtocol}
\author{Wolfgang Maier}
%%
%% Layout
%%
\setlength{\parindent}{0mm}
\setlength{\parskip}{0.5ex plus0.2ex minus0.1ex}
\geometry{lmargin=35mm,rmargin=25mm,tmargin=25mm,bmargin=25mm}
\addtolength{\headheight}{2.5pt}
\pagestyle{fancy}
\renewcommand{\sectionmark}[1]{\markboth{\thesection.~#1}{}}
%\setlength{\headrulewidth}{0pt}
\rhead{\thepage}
\cfoot{}
%%
%% Commands
%%
\newcommand{\FlowProtocol}{\textit{FlowProtocol}\ }
\newcounter{bspcount}
\newcommand{\Beispiel}{\stepcounter{bspcount}\subsubsection*{Beispiel~\arabic{bspcount}}}
\begin{document}
\maketitle
\tableofcontents
\newpage
\section{Vorwort}

% Die Vorteile im Entwicklungsumfeld
%%% Vielseitige Einsatzm�glichkeiten
%%% F�rderung von Teamzusammenarbeit und Prozesskultur
%%% Aussch�pfung von Fachwissen
%%% Zusammenspiel mit anderen Systemen

% Bewertungen
%%% Die Wichtigkeit von Objektivit�t
%%% Arbeiten mit Kriterien
%%% Rangfolgen und das gute alte Bauchgef�hl
%%% Umfragen

% Kleine Expertensysteme
%%% Zielsetzung und Anwendungsbeispiele
%%% Wissen verf�gbar machen ohne Wissensaustausch
%%% Der steinige Weg zum guten Expertensystem

% Das schwere Los des Product Owners
%%% Es allen Leuten Recht getan\dots
%%% Tellerfertige Stories f�r Entwickler
%%% Die Kunst nix zu vergessen
%%% Die Sekret�rIn im Browser

% Die Einbeziehung der Teams
%%% Durchg�ngigkeit beginnt bei der Story
%%% Bereitstellung von Codefragementen
%%% Synergieen mit BestPractice und CleanCode-Prinzipien
%%% Risikobewertung und Risikovermeidung
%%% Unterst�tzung bei Integrations- und Unittests

% Schnell gebaute Werkzeuge
%%% Das gro�e Feld der Textbausteine
%%% Einbeziehung von Fallunterscheidungen
%%% Navigationsb�ume
%%% Anwendungsf�lle f�r generierte Links

\end{document}